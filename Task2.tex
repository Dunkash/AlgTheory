\documentclass[fleqn]{article}
\usepackage[T1]{fontenc}

\usepackage[T2A]{fontenc}
\usepackage[utf8]{inputenc}
\usepackage[russian]{babel}
\usepackage{mathtools}

\usepackage{hyphenat}
\hyphenation{ма-те-ма-ти-ка вос-ста-нав-ли-вать}


\title{Задание 1.2}
\author{Максим Цыба}
\date{\today}



\begin{document}

\maketitle

\textbf{з)}Умножение числа, заданного в унарной системе счисления, на 2.
\begin{gather*}
\Sigma=\{1\}\\
\Sigma'=\Sigma\cup\{*\}
\end{gather*}

\begin{cases}
    *1 \rightarrow 11*\\
    * \rightarrow .\epsilon\\
    \epsilon \rightarrow *
\end{cases}
\[
\_11\Rightarrow \underline{*1} 1 \Rightarrow 11\underline{*1}\Rightarrow 1111\underline{*}\Rightarrow1111
\]
\[
\underline{\epsilon} \Rightarrow \underline{*} \Rightarrow \epsilon
\]

\textbf{и)}Вычисление частного и остатка от деления числа, заданного в унарной системе счисления, на два (над алфавитом \{1,\#\}). Результат должен записываться в виде частное\#остаток. Ноль должен соответствовать пустому слову.
\begin{gather*}
\Sigma=\{1,\#\}\\
\Sigma'=\Sigma\cup\{*\}
\end{gather*}

\begin{cases}
    *11 \rightarrow 1*\\
    * \rightarrow .\#\\
    \epsilon \rightarrow *
\end{cases}
\[
\_1111\Rightarrow \underline{*11}11\Rightarrow 1\underline{*11}\Rightarrow 11\underline{*}\Rightarrow 11\#
\]
\[
\_111\Rightarrow \underline{*11}1\Rightarrow 1\underline{*}1 \Rightarrow 1\#1
\]

\textbf{к)}Дублирование всех символов входного слова (над алфавитом \{a,b\}).
\begin{gather*}
\Sigma=\{\text{a,b}\}\\
\Sigma'=\Sigma\cup\{*\}
\end{gather*}


\begin{cases}
    *a \rightarrow aa*\\
     *b \rightarrow bb*\\
    * \rightarrow .\varepsilon\\
    \varepsilon \rightarrow *
\end{cases}
\[
\underline{\epsilon} \Rightarrow \underline{*} \Rightarrow \epsilon
\]
\[
\_bab \Rightarrow \underline{*b}ab \Rightarrow bb\underline{*a}b\Rightarrow bbaa\underline{*b}\Rightarrow bbaabb\underline{*}\Rightarrow bbaabb
\]

\textbf{л)}Перестановка символов входного слова в обратном порядке (над алфавитом \{a,b\})
\begin{gather*}
\Sigma=\{\text{a,b}\}\\
\Sigma'=\Sigma\cup\{*,\#,|\}
\end{gather*}

\begin{cases}
*a\rightarrow a*\\
*b\rightarrow b*\\
* \rightarrow \# \\
ab\# \rightarrow b\#a\\
aa\# \rightarrow a\#a\\
ba\# \rightarrow a\#b\\
bb\# \rightarrow b\#b\\
|a\# \rightarrow a|*\\
|b\# \rightarrow b|a\\
|\# \rightarrow. \varepsilon\\
\varepsilon \rightarrow |*
\end{cases}
\[
\_ab \Rightarrow |\underline{*a}b \Rightarrow |a\underline{*b} \Rightarrow |ab\underline{*} \Rightarrow |\underline{ab\#} \Rightarrow \underline{|b\#}a \Rightarrow b|\underline{*a} \Rightarrow b|a\underline{*} \Rightarrow  b\underline{|a\#} \Rightarrow ba\underline{|*} \Rightarrow ba 
\]
\[
\underline{\varepsilon} \Rightarrow |\underline {*} \Rightarrow  \underline{|\#} \Rightarrow \varepsilon
\]

\textbf{м)}Cортировка символов входного слова (над алфавитом \{a,b,c\})
\begin{gather*}
\Sigma=\{\text{a,b}\}\\
\Sigma'=\Sigma
\end{gather*}
\begin{cases}
    ba \rightarrow ab\\
    cb \rightarrow bc\\
    ca \rightarrow ac\\
\end{cases}
\[
    \epsilon
\]
\[
    c\underline{ba} \Rightarrow \underline{ca}b \Rightarrow a\underline{cb}\Rightarrow abc
\]

\textbf{н)}Проверка, является ли входное слово палиндромом (над алфавитом \{a,b\}).Если является, то результатом должно быть пустое слово, если не является, то результатом может быть любое непустое слово.
\begin{gather*}
\Sigma=\{\text{a,b}\}\\
\Sigma'=\Sigma\cup\{*,\#,|\}
\end{gather*}

\begin{cases}
*a\rightarrow a*\\
*b\rightarrow b*\\
|a* \rightarrow .\varepsilon\\
|b* \rightarrow .\varepsilon\\
* \rightarrow \# \\
ab\# \rightarrow b\#a\\
aa\# \rightarrow a\#a\\
ba\# \rightarrow a\#b\\
bb\# \rightarrow b\#b\\
|a\#a \rightarrow |*\\
|b\#b \rightarrow |*\\
|a\#b \rightarrow .a\\
|b\#a \rightarrow .b\\
|\# \rightarrow. \varepsilon\\
\varepsilon \rightarrow |*
\end{cases}

\[
\_aa \Rightarrow |\underline{*a}a \Rightarrow |a\underline{*a} \Rightarrow |aa\underline{*} \Rightarrow |\underline{aa\#} \Rightarrow \underline{|a\#a} \Rightarrow |\underline{*} \Rightarrow \underline{|\#} \Rightarrow \varepsilon
\]
\[
\_ab \Rightarrow |\underline{*a}b \Rightarrow |a\underline{*b} \Rightarrow |ab\underline{*} \Rightarrow |\underline{ab\#} \Rightarrow \underline{|b\#a} \Rightarrow b
\]

\textbf{н)}Проверка, является ли входное слово именем одного из основных регистровпроцессора Intel 8088 (AX, BX, CX или DX). Результатом должно быть либо имя регистра, либо пустое слово.
\begin{gather*}
\Sigma=\{\text{A,B,C,D,H,L}\}\\
\Sigma'=\Sigma\cup\{*,\#,|,\$\}\\
\text{Условимся, что Х - любой символ алфавита}
\end{gather*}
\begin{cases}
    *A \rightarrow A\#\\
    *B \rightarrow B\#\\
    *C \rightarrow C\#\\
    *D \rightarrow D\#\\
    * \rightarrow | \\
    \#H \rightarrow H\$\\
    \#L \rightarrow L\$\\
    \# \rightarrow | \\ 
    \$X\rightarrow | \\
    \$ \rightarrow .\varepsilon \\
    X| \rightarrow | \\
    |X \rightarrow | \\
    | \rightarrow .\varepsilon\\
    \epsilon \rightarrow *
\end{cases}

\[
\_AH \Rightarrow \underline{*A}H \Rightarrow A\underline{\#H}\Rightarrow AH\underline{\$}\Rightarrow AH
\]
\[
\_LLL \Rightarrow \underline{*}LLL \Rightarrow \underline{|L}LL\Rightarrow \underline{|L}L\Rightarrow \underline{|L}\Rightarrow \underline{|}\Rightarrow \varepsilon
\]
\end{document}
